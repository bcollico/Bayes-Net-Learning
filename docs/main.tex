\documentclass[twoside,11pt]{article}

\usepackage{aa228-jmlr2e}
\usepackage{lipsum}
\usepackage{listings}

\input{julia_listing}

\begin{document}

% Refer to this link for project rubric: https://web.stanford.edu/class/aa228/cgi-bin/wp/project-1/
\title{Project 1: Bayesian Structure Learning}

\name{Bradley Collicott}
\email{collicott@stanford.edu}


\maketitle


\section{Algorithm Description}
%===========================================
% TODO: Replace this with a short description of your algorithm(s) used.
\lipsum[2]
%===========================================



\section{Graphs}
%===========================================
% TODO: Add your small, medium, and large graph visualizations here
%===========================================
\begin{figure}[h]
    \centering
    \caption{Graph caption.}
\end{figure}



\section{Code}
%===========================================
% TODO: Add your code here, see code listing options here: https://www.overleaf.com/learn/latex/code_listing
% NOTE: Code does not count towards your page limit!
% OPTIONS:
%   1. Paste everything into a {verbatim} environment (where all characters are parsed...verbatim).
%   or 2. paste everything into a {lstlisting} environment for syntax highlighting (examples for Julia and Python below).
% NOTE: Feel free to break up functions into separate {algorithm} + {lstlisting} environments for better organization (not required!) 
%===========================================


%===========================================
% EXAMPLE JULIA: TODO REPLACE WITH YOUR CODE
%===========================================
\begin{algorithm}
\begin{lstlisting}[language=Julia]
using LightGraphs
using Printf

"""
    write_gph(dag::DiGraph, idx2names, filename)

Takes a DiGraph, a Dict of index to names and a output filename to write the graph in `gph` format.
"""
function write_gph(dag::DiGraph, idx2names, filename)
    open(filename, "w") do io
        for edge in edges(dag)
            @printf(io, "%s,%s\n", idx2names[src(edge)], idx2names[dst(edge)])
        end
    end
end


function compute(infile, outfile)

    # WRITE YOUR CODE HERE
    # FEEL FREE TO CHANGE ANYTHING ANYWHERE IN THE CODE
    # THIS INCLUDES CHANGING THE FUNCTION NAMES, MAKING THE CODE MODULAR, BASICALLY ANYTHING

end

if length(ARGS) != 2
    error("usage: julia project1.jl <infile>.csv <outfile>.gph")
end

inputfilename = ARGS[1]
outputfilename = ARGS[2]

compute(inputfilename, outputfilename)
\end{lstlisting}
\end{algorithm}

\end{document}
